\documentclass{article}
\renewcommand{\familydefault}{\sfdefault}
\usepackage{graphicx} % Required for inserting images
\usepackage{helvet}
\usepackage{geometry}
\usepackage{hyperref}
\usepackage{listings}
\geometry{top=2.5cm, bottom=2cm}
\usepackage{multirow}
\usepackage{titlesec}

\setcounter{secnumdepth}{4}


\begin{document}

\begin{figure}[h]
    \centering
    \includegraphics[width=10cm]{images/Logo.png}
\end{figure}

\title{Quanto-Solutions}

\date{November 2023}

\renewcommand{\listfigurename}{Abbildungsverzeichnis}

\maketitle

\begin{table}[h]
    \centering

    \begin{tabular}{|p{3cm}|c|p{3cm}|}
        \hline
        \textbf{Name}    & \textbf{Matrikelnummer} & \textbf{Githubaccount} \\
        \hline
        Chris Schröder   & 767384                  & Chris0297              \\
        Michele Pomarico & 766583                  & Michele-92             \\
        Fabian Sonek     & 767903                  & e2Neo                  \\
        Baran Kal        & 769042                  & Roplck                 \\
        Daniel Ryssel    & 768960                  & danGitRys              \\
        \hline
    \end{tabular}
\end{table}

\newpage

\tableofcontents  %Inhaltsverzeichnis

\newpage
\listoffigures  % Abbildungsverzeichnis
\newpage

\section{Introduction}

\section{Kurzfassung}
Dieses Dokument behandelt die Dokumentation der Projektmanagement Software,
welche im Rahmen des Moduls “Projekt Softwaretechnik und Medieninformatik” im
Studiengang Softwaretechnik und Medieninformatik an der Hochschule Esslingen
für die Firma Quanto-Solutions entwickelt wird.

\newpage

\section{Zielgruppe/Problemstellung}
Zum aktuellen Zeitpunkt werden von der Firma Quanto-Solutions für die
Projektverwaltung Excel-Tabellen verwendet, um die Mitarbeiter einem Projekt
zuzuordnen und die Arbeitszeit zu erfassen. Dabei gibt es das Problem des
Verwaltens der Zugriffsrechte auf die Excel Dateien, da aus Datenschutzgründen
nicht jeder Mitarbeiter alle Informationen einsehen darf, was darin resultiert,
dass nur die Managementebene in der Firma diese nutzen darf. Dadurch gestaltet
sich das Eintragen der Arbeitsstunden als sehr Zeit aufwändig. Um diese
Probleme zu lösen, wurden wir beauftragt, eine Projektmanagement-Software zu
entwickeln, die für unsere Zielgruppe, alle Mitarbeiter von Quanto-Solutions,
zugänglich ist und eine Möglichkeit bietet, IST-Aufwände schnell und
komfortabel buchen zu können.

\subsection{ User Stories}
\subsubsection{Nutzer (Mitarbeiter, Projekt Manager, Manager) }
\begin{itemize}
    \item Als Nutzer möchte ich mich einloggen können, um Zugriff auf das
          Projektmanagement-Tool zu erhalten. (Login)
    \item Als Nutzer möchte ich meine Arbeitszeiten erfassen können, um diese auf ein
          Projekt zu buchen. (Time Registration)
    \item Als Nutzer möchte ich meine eingeplanten Arbeitszeiten sehen können, um einen
          Überblick über meine erfassten und geplanten Arbeitszeiten zu bekommen.
          (Working Times)
    \item Als Nutzer möchte ich meine Arbeitszeiten korrigieren können, um eine korrekte
          Erstellung der Rechnung zu gewährleisten. (Time Correction)
    \item Als Nutzer möchte ich mit Hilfe der Sidebar schnell die Ansichten wechseln
          können, um effektiver zu arbeiten. (Sidebar)
    \item Als Nutzer möchte ich in meinem Home Screen alle meine Projekte aufgelistet
          bekommen, um eine Übersicht des aktuellen Standes der Projekte zu erlangen.
          (Home)
\end{itemize}

\subsubsection{ Projekt Manager}
\begin{itemize}
    \item Als Projekt Manager möchte ich Projekte verwalten können, um die Mitarbeiter
          auf Tagesebene einem Projekt in einer bestimmten Position zuzuordnen. (Manage
          Project)
    \item Als Projekt Manager möchte ich die Arbeitszeiten der Mitarbeiter sehen können
          und Sie manuell nachtragen, um die richtige Abrechnung der Positionen
          gewährleisten zu können. (Working Times+)
\end{itemize}

\subsubsection{ Manager}
\begin{itemize}
    \item Als Manger möchte ich ein neues Projekt anlegen können mit folgenden
          Informationen: Kunde, Projektname, Zeitraum, Projektmanager, hinzufügen von
          Mitarbeitern, Positionen und Tagessätzen, um einen bessere Projektübersicht und
          damit eine strukturierte Planungsmöglichkeit zu erhalten. (New Project)
    \item Als Manager möchte ich Projekte verwalten können, um die Mitarbeiter auf
          Tagesebene einem Projekt in einer bestimmten Position zuzuordnen. (Manage
          Project)
    \item Als Manager möchte ich die Arbeitszeiten der Mitarbeiter sehen können und Sie
          manuell nachtragen, um die richtige Abrechnung der Positionen gewährleisten zu
          können. (Working Times+)
\end{itemize}

\subsection{ Functional Requirements}

\subsubsection{Projektanlage für Manager}
Das System muss es Managern ermöglichen, neue Projekte anzulegen, wobei sie
wichtige Informationen wie Kundenname, Projektbezeichnung und den Zeitraum
festlegen können. Dies erleichtert die klare Definition und Überwachung der
Projekte von ihrem Beginn an.
\subsubsection{Projektmanagemnt-Befugnisse für Manager}
Manager sollten in der Lage sein, für jedes Projekt einen verantwortlichen
Projektmanager (PM) zu bestimmen und Mitarbeiter für das Projekt zuzuordnen
sowie den Mitarbeitern Positionen zuordnen zu können. Diese Funktion
gewährleistet, dass die Zuständigkeiten klar verteilt sind und die Teams gut
organisiert sind.
\subsubsection{Tägliche Einsatzplanung für Projektmanager}
Projektmanager müssen auf Tagesebene eine detaillierte Einsatzplanung für die
Mitarbeiter erstellen können. Dies ermöglicht eine präzise und effektive
Zeiterfassung, um sicherzustellen, dass Projekte termingerecht abgeschlossen
werden.
\subsubsection{Zeitaufzeichung und Reminder für Mitarbeiter}
Die Mitarbeiter sollten in der Lage sein, ihre “IST-Aufwände” zu erfassen, um
den Arbeitsfortschritt und die Aufgabenverwaltung zu gewährleisten. Darüber
hinaus sollte das System automatische Erinnerungen bereitstellen, um
sicherzustellen, dass die Zeitaufzeichnung zeitnah und korrekt erfolgt.
\subsubsection{Berechtigungsabhängige Funktionalitäten}
Die Funktionalitäten des Systems sollten abhängig von der Benutzerrolle
variieren. Das bedeutet, dass Manager, Projektmanager und Mitarbeiter jeweils
nur auf die Funktionen und Informationen zugreifen können, die für ihre
jeweiligen Aufgaben relevant sind. Dies gewährleistet die Sicherheit von Daten
und erhöht die Effizienz, da Benutzer nicht mit irrelevanten Informationen
überlastet werden.
\subsection{ Nonfunctional Requirements}
Als Non-Functional Requirements in diesem Projekt ist wichtig, dass die
Anwendung als Web-Anwendung läuft.

\newpage

\section{High Level Arichtektur Programm}
\subsection{ Struktursicht}
\begin{figure}[h]
    \centering
    \includegraphics[scale=0.85]{images/Struktursicht.png}
    \caption{Komponentendiagramm}
    \label{fig:beispiel}
\end{figure}
\newpage
\subsection{ Verhaltenssicht}
\begin{figure}[h]
    \centering
    \includegraphics[scale=0.9]{images/Verhaltenssicht.png}
    \caption{Ablaufdiagramm}
    \label{fig:beispiel}
\end{figure}
\newpage
\subsection{ Verteilungssicht}
\begin{figure}[h]

    \includegraphics[width= \textwidth]{images/Verteilungssicht.png}
    \caption{Verteilungsdiagramm}
    \label{fig:beispiel}
\end{figure}

\newpage

\section{Architektur/ Technologie}

\subsection{Github}
Zur Versionsverwaltung des Codes des Projektes wird Github verwendet. Darüber
hinaus bietet GitHub Werkzeuge zur Fehlerverfolgung an, die es uns ermöglichen,
Probleme und Verbesserungen in unserem Projekt zu dokumentieren und
nachzuverfolgen. Die Verwendung von GitHub erleichtert die Zusammenarbeit und
ermöglicht eine transparente und gut organisierte Entwicklungsumgebung für das
Projekt.

\begin{figure}[h]
    \centering
    \includegraphics[width= \textwidth]{images/GitHub.png}
    \caption{Screenshot Github}
    \label{fig:beispiel}
\end{figure}
\subsection{Genutzte IDEs}
Als IDE wird VS-Code von Microsoft verwendet. Aufgrund der Konfigurierbarkeit
von VS-Code wird es für die Entwicklung des Front- und Backend verwendet.

\begin{figure}[h]
    \centering
    \includegraphics[width= \textwidth]{images/VS-Code.png}
    \caption{Screenshot VS-Code}
    \label{fig:beispiel}
\end{figure}
\subsection{Postman}
Zum Testen der Backend API wird Postman genutzt. Postman ist sowohl als lokale
Anwendung nutzbar, als auch als Webanwendung oder als Plug-in VS-Code. Postman
ermöglicht es dazu noch das man seine Collections in einem Team zusammen
bearbeiten und teilen kann

\begin{figure}[h]
    \centering
    \includegraphics[width= \textwidth]{images/Postman.png}
    \caption{Screenshot Postman}
    \label{fig:beispiel}
\end{figure}
\subsection{Figma}
Zum Erstellen unserer Mockups wird die Software Figma genutzt, da diese es
ermöglicht, im Team gleichzeitig an Entwürfen zu arbeiten. Als erstes gab es
die Entscheidung, welche Farbe für das Mockup genutzt werden sollte. Dabei
hatten wir die Wahl zwischen einem orangen oder blauen Mockup, da diese sehr
gut zum Stil von Quanto Solutions passen. Nach verschiedenen Tests haben wir
uns entschieden, ein blaues Mockup zu erstellen. Die verschiedenen Screens
werden im Folgenden gezeigt und näher erläutert. Der Login-Screen (Abbildung ),
wo sich der Nutzer mit der Mitarbeiternummer und einem selbst gewählten
Passwort einloggen kann, dient dazu, dass keine Person außerhalb der Firma
Zugang erhält.

\subsection{Microsoft SQL Server Managment Studio}
Zum Verwalten, Interagieren und Entwickeln unserer Datenbank nutzen wir
Microsoft SQL Server Management Studio, da dieses Programm von Microsoft selbst
stammt und wir sowohl die Datenbank als auch die Verwaltungssoftware für diese
aus einer Hand haben.

\begin{figure}[h]
    \centering
    \includegraphics[width= \textwidth]{images/Microsoft SQL Server Management Studio.png}
    \caption{Screenshot Micorosoft SQL Server Managment Studio}
    \label{fig:beispiel}
\end{figure}
\subsection{Frontend: Vue.js}
\subsection{Datenbank: Microsoft SQL Server}
\subsection{Authentifizierung SAP}
\subsection{Projektmanagment Jira}
Warum wir Jira als Kommunikationstool nutzen:

Wir setzen Jira als unser Kommunikationstool in unser Softwareprojekt ein, weil
es sich als äußerst effizientes und vielseitiges Werkzeug für die interne und
externe Kommunikation erwiesen hat. Wir nutzen Jira, um Informationen,
Aufgaben, Fehler und Anforderungen in unseren Projekten zentral zu verwalten,
was die Klarheit und Effizienz in der Projektarbeit fördert.
\begin{enumerate}
    \item In Jira können wir in Echtzeit zusammenarbeiten und auf Aktualisierungen sowie
          Kommentare zugreifen, was eine schnelle und nahtlose Kommunikation unabhängig
          von den Standorten unserer Teammitglieder ermöglicht.
    \item Die Anpassungsfähigkeit von Jira erlaubt es uns, Arbeitsabläufe und Prozesse
          exakt auf unsere Projektanforderungen zuzuschneiden und benutzerdefinierte
          Felder hinzuzufügen, um projektbezogene Informationen effizient zu verfolgen.
    \item Die Verwendung von Jira steigert die Transparenz und Nachverfolgbarkeit in
          unseren Projekten, indem alle Teammitglieder einen klaren Überblick über den
          aktuellen Stand und den Fortschritt der Aufgaben haben.

\end{enumerate}
Vorteile unsere Vorgehensweise mit Jira
\begin{enumerate}
    \item Jira ermöglicht eine effiziente Planung und Priorisierung von Aufgaben und
          Anforderungen, indem wir Arbeitsabläufe und Sprints definieren und den
          Fortschritt leicht im Blick behalten.
    \item Es erleichtert die detaillierte Meldung und Verfolgung von Softwarefehlern, was
          die Qualität unserer Software maßgeblich steigert.
    \item Wir können Aufgaben klar zuweisen und Verantwortlichkeiten definieren, um
          sicherzustellen, dass jedes Teammitglied weiß, welche Aufgaben es zu erfüllen
          hat.
    \item Jira fördert die Echtzeitkommunikation und den Austausch von Kommentaren zu
          Aufgaben und Anforderungen, was die Zusammenarbeit und den reibungslosen
          Informationsfluss im Team unterstützt. \item Die umfassende Rückverfolgbarkeit von Anforderungen, Aufgaben und Problemen in
          Jira ist von großer Bedeutung in der Softwareentwicklung, um sicherzustellen,
          dass alle Anforderungen vollständig und termingerecht erfüllt werden.
\end{enumerate}

\newpage

\section{Aufwandsschätzung}
Da dieses Projekt im Rahmen des Moduls “Projekt Softwaretechnik und
Medieninformatik” im Studiengang Softwaretechnik und Medieninformatik an der
Hochschule Esslingen stattfindet, gibt es einen vorgeschriebenen Aufwand von 10
ECTS pro Person was 300 Stunden entspricht. Da das Team, welches dieses Projekt
umsetzt, aus 5 Personen besteht, gibt es insgesamt 1500 Stunden, die in diesem
Projekt verteilt werden können. Wir messen den Aufwand in Personentagen (PT),
welche jeweils 8 Arbeitsstunden beinhalten. Dadurch kommen wir auf ein Gesamt
Pensum von 187,5 PT, welches von der Hochschule vorgeschrieben ist. Bei der
Methode der Aufwandsschätzung haben wir uns für die Bottom-Up-Methode
entschieden. Hierbei arbeiten wir die Anforderungen des Projekts durch und
schätzen für jede den entsprechenden Arbeitsaufwand.

Die Aufwandsschätzung verteilt sich bei diesem Projekt dabei folgendermaßen:

Um die einzelnen Anforderungen abzuarbeiten, brauchen wir erstmal ein
Grundgerüst unserer Software, auf dem wir dann aufbauen können.

\begin{enumerate}
    \item Frontend anlegen: 5PT
    \item Backend anlegen: 5PT
    \item Datenbank anlgegen: 5PT
    \item Prototyp eines Login-Systems implementieren und die Anforderung: “Die
          vorgesehenen Funktionalitäten für Manager PM und Mitarbeiter müssen jeweils nur
          für die jeweils nur für die jeweiligen Personen zur Verfügung stehen”: 5PT
\end{enumerate}

Im folgenden werden die Anforderungen geschätzt die auf dem Grundgerüst
aufbauen

\begin{enumerate}
    \item SAP Login einrichten: 15PT
    \item Manager müssen Projekte anlegen können (Kunde, Projektname, Budget, Zeitraum):
          10PT
    \item Manager müssen für Projekte Projektmanager (PM) festlegen und Mitarbeiter
          zuordnen können: 15PT
    \item PM müssen auf Tagesebene eine zeitliche Einsatzplanung vornehmen und
          Mitarbeiter zuordnen können: 30PT
    \item Mitarbeiter müssen ihre IST-Aufwände buchen können. Zudem sollen Mitarbeiter
          dafür einen Reminder erhalten: 30PT
    \item Tagessätze Vor Ort und Remote unterscheiden sich! Jeweils eine
          Rechnungsposition: 5PT
    \item Arbeitspensum wird jeweils für Vor-Ort und Remote getrennt festgelegt: 5PT
    \item Es soll automatisiert eine Rechnung erzeugt werden und per E-Mail an die
          Rechnungsstellung geschickt werden: 10PT
    \item Puffer zm flexibel einteilen: 27,5 PT

\end{enumerate}

\newpage

\section{Mockup}
\subsection{LoginScreen}

Über den Login-Screen kann der Nutzer sich mit seinem SAP-BTP Account anmelden. Es gibt 2 Eingabefelder, dass eine frägt nach der E-Mail Adresse des Nutzer das andere nach dem Passwort.
Wenn der Login erfolgreich war wird der Nutzer auf den Homescreen weitergeleitet. Falls nicht kommt ein Alert mit der Nachricht "Invalid Login" (siehe Abbildung 8).

\begin{figure}[h]
    \includegraphics[height= 0.5\textwidth,width= \textwidth]{images/Login.png}
    \caption{Login-Ansicht}
    \label{fig:beispiel}
\end{figure}

\subsection{Homescreen}
Nach dem Login wird man zum Homescreen (siehe Abbildung 9) weitergeleitet. Auf
dem Homescreen werden alle Projekte angezeigt, in denen man involviert ist,
sowie die jeweiligen Rollen in dem Projekt. Wenn man ein Projekt auswählt, in
dem man Projektleiter ist, ändert sich die Sidebar und man bekommt mehr
Zugriffsrechte und somit tiefere Einblicke in das Projekt. Dabei sieht die
Sidebar je nach Rolle unterschiedlich aus. Der Manager, Projektleiter und
Mitarbeiter haben jeweils unterschiedliche Zugriffsrechte. Um die Ansichten zu
wechseln, kann man die Sidebar oder auch die orangen Pfeile benutzen. Die
Rollen des Managers, Projektleiters und des Mitarbeiters sehen im Homescreen
den Projektnamen, den Projektleiter des Projektes sowie das Datum des
Projektendes. Der Projektleiter und der Mitarbeiter sehen dazu noch die Rolle,
die ihnen in den Projekten zugeteilt wurde. Der Manager kann alle Projekte sehen
die es aktuell gibt und hat auch auf jedes Projekt Zugriff.

\begin{figure}[h]
    \includegraphics[height= 0.5\textwidth,width= \textwidth]{images/Home.png}
    \caption{Home-Ansicht}
    \label{fig:beispiel}
\end{figure}

\newpage
\subsection{New Project}
Über die Eingabemaske New Project kann der Manager ein ganz neues Projekt anlegen. Alle Eingaben erfolgen über ein Eingabefeld. In der Maske kann er alle Daten, die für ein Projekt relevant sind, eingeben, die da wären: Projektnamen, Kunden-Namen, Projektleiter hinzufügen, Projektstart und  Projektende. (Abbildung 10). 
Zu dem kann er über ein DropDown Menü Mitarbeiter auswählen, die
er diesem Porjekt zuweisen möchte. Über ein Eingabefeld können neue
Position hinzugefügt werden, welche jeweils mit einem Tagessatz versehen werden.
Diese Positionen können den Mitarbeitern dann zugeteil werden, ein Mitarbeiter
kann mehrere Position für ein Projekt haben.Wenn die Position einem Mitarbeiter 
zugeteilt wurden muss man noch die Personentage (PT) angeben. Anhand
der Daten von Personentage und Tagessatz wird das Gesamt Budget für das 
Projekt errechnet. Der Vorgang kann über ein Submit Button abgeschlossen werden.


\begin{figure}[h]
    \includegraphics[height= 0.5\textwidth,width= \textwidth]{images/New Project.png}
    \caption{New Project Ansicht}
    \label{fig:beispiel}
\end{figure}

\begin{figure}[h]
    \includegraphics[height= 0.5\textwidth,width= \textwidth]{images/Positionen einfügen.png}
    \caption{Position-Einfügen Ansicht}
    \label{fig:beispiel}
\end{figure}

\subsection{Manage Project}
In der Ansicht “Manage Project” wird ein ausgewähltes Projekt in einer Tabelle
angezeigt. In der Tabelle sieht man für einen über das Dropdown-Menü ausgewählten
Monat alle Arbeitszeiten, der in diesem Projekt involvierten Mitarbeiter.
In der Spalte Summe Projects wird die Stundenanzahl der summierten Projekte
angezeigt in dem der Mitarbeiter an diesem Tag bereits eingeplant wurde.
Somit bekommt man einen Überblick welcher Mitarbeiter an welchem Arbeitstag 
noch wie viel Arbeitsstunden zur Verfügung hat. Über die Spalte This Project
kann man nun die Anzahl der Stunden eingeben für die man den Mitarbeiter
an dem Tag einplanen möchte. In der Spalte Pos (Positions) wählt man aus 
in welcher Position der Mitarbeiter an diesem Tag eingeplant werden soll
(siehe Abbildung 12).


\begin{figure}[h]
    \includegraphics[height= 0.5\textwidth,width= \textwidth]{images/ManageProject1.png}
    \caption{Manage-Project-Ansicht}
    \label{fig:beispiel}
\end{figure}

\newpage

\subsection{Sidebar}
Mit Hilfe der Sidebar kann man schnell von einer Ansicht auf eine andere kommen.
Die Sidebar ist je nach Rolle unterschiedlich strukturiert. Der Mitarbeiter hat
eine andere Ansicht als der Projektleiter. Jede Rolle hat nichtsdestotrotz die
folgenden Buttons: Den “Home” Button, der wie zuvor gezeigt, den Homescreen
anzeigt, den “Working-Times” Button, der die gearbeiteten Zeiten in Form eines
Kalenders angezeigt, die Time Registration, in der man seine gearbeiteten
Zeiten eingepflegt und die Einstellungen. Der Manager hat zudem noch einen
New-Project-Button, mit dem neue Projekte erstellt werden können (siehe Abbildung
10). Der Projektleiter hat einen Manage-Project-Button, auf dieser Ansicht kann
er die Mitarbeiter auf Stunden Ebene einem Projekt zuordnen (siehe Abbildung 12).
Mehr Details folgen bei den dazugehörigen Screenshots (siehe Abbildung Sidebar 13).

\begin{figure}[h]
    \includegraphics[height= 0.5\textwidth,width= \textwidth]{images/Sidebar.png}
    \caption{Sidebar-Ansicht}
    \label{fig:beispiel}
\end{figure}

\newpage
\subsection{Time Registration}
Über das Fenster “Time Registration” können die Mitarbeiter Ihre Arbeitszeiten erfassen.  Als Erstes geben Sie die Startzeit ein, also um wie viel Uhr haben Sie angefangen zu arbeiten. 
Dann seine Pausenzeiten und Endzeit. Über ein DropDown Menü hat man noch die Möglichkeit, sich die Projekte anzeigen zu lassen, in denen man involviert ist und auf das man seine Zeit erfassen möchte. 
Wichtig ist auch noch, dass die richtige Position ausgewählt wird, da bei jeder Position ein anderer Verrechnungssatz hinterlegt ist (siehe Abbildung 14).
\begin{figure}[h]
    \includegraphics[height= 0.5\textwidth,width= \textwidth]{images/Time Registration.png}
    \caption{Time Registration Ansicht}
    \label{fig:beispiel}
\end{figure}

\subsection{Working Times / Working Times+}
In der Ansicht “Working Times” sieht der Mitarbeiter seine Arbeitszeiten in
einer Übersicht aufgeschlüsselt. Die Zeiten werden in eingeplanten
Arbeitsstunden pro Tag angezeigt. Es wird angezeigt, wie lange der Mitarbeiter
tatsächlich gearbeitet hat, sowie seine Pausenzeiten. In der letzten Zeile der
Tabelle sieht er die Summe der geplanten Arbeitsstunden und tatsächlichen
Arbeitsstunden. Dadurch weiß der Mitarbeiter, wie weit er in dem Projekt ist
und wie viel Zeit er an diesem Tag dafür verwendet hat (siehe Abbildung 15). \\
In der Ansicht Working Times+ sieht der Projektleiter die Arbeitszeiten 
seiner Mitarbeiter die diese über die Anischt Time Correction (siehe Abbildung xx)
kontrolliert und bestätigt haben. Per DropDown Menü kann man zwischen Projekten und
Mitarbeitern filtern somit bekommt man eine bessere Übersicht, falls man nach etwas
bestimmten sucht. Der Projektleiter kann in dieser Ansicht noch Nachtragungen von
Arbeitszeiten der Mitarbeiter erfassen, falls diese bei Ihrer Zeiterfassung
einen Fehler gemacht hätten (siehe Abbildung 16).


\begin{figure}[]
    \includegraphics[height= 0.5\textwidth,width= \textwidth]{images/Working Times.png}
    \caption{Working Times Ansicht}
    \label{fig:beispiel}
\end{figure}

\begin{figure}[]
    \includegraphics[height= 0.5\textwidth,width= \textwidth]{images/Working Times+.png}
    \caption{Working Times+ Ansicht}
    \label{fig:beispiel}
\end{figure}

\newpage
\clearpage

\subsection{Time Correction}
In der Ansicht Time Correction sieht der Mitarbeiter alle seine Arbeitszeiten
von seine Porjekten in einer Tabelle aufgelistet, in dennen er eingeplant war
aber seine Zeit nicht erfasst hat. Er hat die Möglichkeit über die Worked Spalte
nachzutragen (siehe Abbildung 17). Wenn der Mitarbeiter am Monatsende seine
Arbeitszeiten kontrolliert hat und alle Zeiten erfasst wurden, kann er auf 
den Confirm Button drücken. Dann werden die Arbeitszeiten bestätigt und landen
beim Projektleiter in der Working Times+ Anischt (siehe Abbildung 16).


\begin{figure}[]
    \includegraphics[height= 0.5\textwidth,width= \textwidth]{images/TimeCorrection.png}
    \caption{Working Times+ Ansicht}
    \label{fig:beispiel}
\end{figure}




\subsection{Create New Employee}
In der Ansicht Create New Employee kann der Manager einen neuen Mitarbeiter
anlegen. Hierbei werden folgende Informationen benötigt: Vorname,Nachname,E-Mail Adresse,
Telefonnummer und Teamrolle. 

\begin{figure}[h]
    \centering
    \includegraphics[width= \textwidth]{images/CreateNewEmployee.png}
    \caption{Create New Employee Ansicht}
    \label{fig:beispiel}
\end{figure}

\newpage
\clearpage

\section{Datenbank}
Test e
\begin{figure}[h]
    \centering
    \includegraphics[width= \textwidth]{images/logischeAnsicht.png}
    \caption{Logische Sicht Datenbank}
    \label{fig:beispiel}
\end{figure}

\begin{figure}[h]
    \centering
    \includegraphics[width= \textwidth]{images/datenbankphysisches-Layout.png}
    \caption{Physisches Layout Datenbank}
    \label{fig:beispiel}
\end{figure}

\newpage

\subsection{Tabellen}

\subsubsection{Employee}
\begin{lstlisting}[language=Sql, caption= Create Table Statement für Employee Table]
    CREATE TABLE [dbo].[employee] (
    [id]         INT           IDENTITY (1, 1) NOT NULL,
    [emp_id]     NVARCHAR (50) NOT NULL,
    [forename]   NVARCHAR (50) NOT NULL,
    [surname]    NVARCHAR (50) NOT NULL,
    [mail]       NVARCHAR (50) NOT NULL,
    [phone]      NVARCHAR (50) NULL,
    [fk_team_id] INT           NULL,
    [team_roll]  NVARCHAR (50) NULL,
    CONSTRAINT [PK_employee] PRIMARY KEY CLUSTERED ([id] ASC),
    CONSTRAINT [FK_employee_team] FOREIGN KEY ([fk_team_id]) REFERENCES 
    [dbo].[team] ([id])
);
        \end{lstlisting}

In der Tabelle Employee werden die Mitarbeiter der Firma gespeichert die sowohl
Projekten zugeordnert werden können, als auch diese administieren.
\paragraph{Id}
Bei der Id handelt es sich um den Primärschlüssel der Tabelle welcher zugleich
ein Autoinkrement-Schlüssel ist, und dazu dient einen Employee eindeutig zu
identifzieren.

\paragraph{emp\_id} Die emp\_id ist eine Id welche einem Mitarbeiter zur Identifizierung
zugeordnert wird. Bei dieser kann es sich um eine exisitierende aus dem
Unternehmen handeln oder auch eine neu implementierte Identifizierungsvariante
handeln. Diese sollte dabei nicht mit der id Spalte in dieser Tabelle
verwechselt werden.

\paragraph{forename,surname und phone} Bei diesen Spalten handelt es sich bei dem Vornamen und Nachname um
erforderliche Informationen und bei der Telefonnummer um eine optionale
Informationen.

\paragraph{mail} Der Mail-Adresse des Employees kommt eine besondere Wichtigkeit zu, da diese
hinsichtlich des Logins mit SAP-BTP entscheidend ist. Die Mail Adresse sollte
hierbei die gleiche sein, die auch in SAP verwendet wird, da die Zuordnung
zwischen SAP und Datenbank über diese erfolgt. In Konsequenz davon sollten die
Mail Adressen in der Datenbank einzigartig sein, da doppelte Mail Addressen zu
Konflikten führen werden.

\paragraph{fk\_team\_id} Bei dem fk\_team\_id handelt es sich um einen Foreign Key mit Bezug auf den Team Table. Durch die Implementierung des Foreign Keys an
dieser Stelle kann ein Employee zu einem Zeitpunkt, nur einem Team zugeordnert sein.

\paragraph{team\_roll}Ein Employee hat in einem Projekt eine Rolle. Diese kann sein "Team-Leader" oder "Member".

\subsubsection{Team}
\begin{lstlisting}[language=Sql, caption= Create Table Statement für Team Table]
    CREATE TABLE [dbo].[team] (
    [id]   INT            IDENTITY (1, 1) NOT NULL,
    [name] NVARCHAR (50)  NOT NULL,
    [info] NVARCHAR (MAX) NULL,
    CONSTRAINT [PK_team] PRIMARY KEY CLUSTERED ([id] ASC)
);

         \end{lstlisting}
\subsubsection{Project}
\begin{lstlisting}[language=Sql, caption= Create Table Statement für Project Table]
    CREATE TABLE [dbo].[project] (
    [id]            INT           IDENTITY (1, 1) NOT NULL,
    [p_id]          NVARCHAR (50) NOT NULL,
    [name]          NVARCHAR (50) NOT NULL,
    [company]       NVARCHAR (50) NOT NULL,
    [start_date]    DATE          NOT NULL,
    [end_date]      DATE          NOT NULL,
    [fk_creator]    INT           NOT NULL,
    [creation_date] DATETIME      NOT NULL,
    CONSTRAINT [PK_project] PRIMARY KEY CLUSTERED ([id] ASC),
    CONSTRAINT [FK_project_creator] FOREIGN KEY ([fk_creator]) 
    REFERENCES [dbo].[employee] ([id])
);
         \end{lstlisting}
\subsubsection{Position}
\begin{lstlisting}[language=Sql, caption= Create Table Statement für Position Table]
    CREATE TABLE [dbo].[position] (
    [id]               INT           IDENTITY (1, 1) NOT NULL,
    [position_id]      NVARCHAR (50) NOT NULL,
    [fk_project]       INT           NOT NULL,
    [rate]             FLOAT (53)    NOT NULL,
    [wd]               FLOAT (53)    NOT NULL,
    [volume_total]     FLOAT (53)    NOT NULL,
    [volume_remaining] FLOAT (53)    NOT NULL,
    [start_date]       DATE          NOT NULL,
    [end_date]         DATE          NOT NULL,
    CONSTRAINT [PK_position] PRIMARY KEY CLUSTERED ([id] ASC),
    CONSTRAINT [FK_position_project] FOREIGN KEY ([fk_project]) 
    REFERENCES [dbo].[project] ([id])
);
         \end{lstlisting}
\subsubsection{Assignment}
\begin{lstlisting}[language=Sql, caption= Create Table Statement für Assignment Table]
    CREATE TABLE [dbo].[assignment] (
    [id]          INT           IDENTITY (1, 1) NOT NULL,
    [fk_project]  INT           NOT NULL,
    [fk_employee] INT           NOT NULL,
    [role]        NVARCHAR (50) NOT NULL,
    CONSTRAINT [PK_assignment] PRIMARY KEY CLUSTERED ([id] ASC),
    CONSTRAINT [FK_assignment_employee] FOREIGN KEY ([fk_employee]) REFERENCES
    [dbo].[employee] ([id]),
    CONSTRAINT [FK_assignment_project] FOREIGN KEY ([fk_project]) REFERENCES
    [dbo].[project] ([id])
    );
    GO
    CREATE UNIQUE NONCLUSTERED INDEX [Index_Assignment_Id]
    ON [dbo].[assignment]([id] ASC);


         \end{lstlisting}
\subsubsection{Booking}
\begin{lstlisting}[language=Sql, caption= Create Table Statement für Booking Table]
    CREATE TABLE [dbo].[booking] (
    [id]          INT        IDENTITY (1, 1) NOT NULL,
    [fk_employee] INT        NOT NULL,
    [fK_position] INT        NOT NULL,
    [start]       DATETIME   NOT NULL,
    [end]         DATETIME   NOT NULL,
    [pause]       FLOAT (53) NULL,
    [time]        INT        NULL,
    CONSTRAINT [PK_tracking] PRIMARY KEY CLUSTERED ([id] ASC),
    CONSTRAINT [FK_tracking_employee] FOREIGN KEY ([fk_employee]) REFERENCES
    [dbo].[employee] ([id]),
    CONSTRAINT [FK_tracking_position] FOREIGN KEY ([fK_position]) REFERENCES
    [dbo].[position] ([id])
);


         \end{lstlisting}
\subsubsection{Forecast}
\begin{lstlisting}[language=Sql, caption= Create Table Statement für Forecast Table]
    CREATE TABLE [dbo].[forecast] (
    [id]          INT            IDENTITY (1, 1) NOT NULL,
    [fk_employee] INT            NOT NULL,
    [fk_position] INT            NOT NULL,
    [start]       DATETIME       NOT NULL,
    [end]         DATETIME       NOT NULL,
    [info]        NVARCHAR (MAX) NOT NULL,
    CONSTRAINT [PK_plan] PRIMARY KEY CLUSTERED ([id] ASC),
    CONSTRAINT [FK_plan_position] FOREIGN KEY ([fk_position]) REFERENCES 
    [dbo].[position] ([id])
);


         \end{lstlisting}

Der SQL Code zum erstellen der Datenbank inklusive der Tabellen, ist im
Github-Repository des Backends zu finden.

\newpage

\section{Schnittstellen Definition}

\subsection{Datenbank Backend}
Für die Schnittstelle zwischen der Datenbank und dem Backend haben wir uns für
die Micosoft ODBC Schnittstelle entschieden. Diese kann wiefolgt beschrieben
werden.

"Die Microsoft Open Database Connectivity (ODBC)-Schnittstelle ist eine C-Programmiersprachenschnittstelle, mit der Anwendungen über eine Vielzahl von Datenbankverwaltungssystemen (DBMSs) auf Daten zugreifen können. ODBC ist eine schnittstelle mit niedriger Leistung, die speziell für relationale Datenspeicher entwickelt wurde."\cite{MSODBC}.

\begin{figure}[h]
    \centering
    \includegraphics[width= \textwidth]{images/odbc.png}
    \caption{Veranschaulichung ODBC Schnittstelle}
    \label{fig:beispiel}
    \cite{ODBC}
\end{figure}

Die ODBC Schnittstelle bietet dabei für uns mehrere Vorteile. Auch wenn in
diesem Projekt als Datenbank Microsoft SQL Server verwendet wird, besteht die
Möglichkeit das auch eine andere Datenbank-Technologie verwendet wird, solange
diese ebenfalls die ODBC Schnittstelle unterstützt. Dadurch vergrößert sich die
potenzielle Zielgruppe unseres Projekts, da für den Fall dass potenzielle
Nutzer andere Datenbank-Systeme präferieren und verwenden wollen. Ein weiterer
Vorteil der sich ergibt ist die Plattformunabhängigkeit, wodurch das Projekt
auf allen gängigen Betriebssystemen laufen kann.

\subsection{Backend Frontend}
Als Schnittstelle zwischen dem Backend und Frontend wird eine REST-Api
verwendet. Dies ist die Konsequenz aus der Entscheidung für das Backend
Framework, da es sich dabei um ein Framework handelt, welches eine REST-API
implementiert, und wir daran nichts ändern, da die REST-API viele Vorteile für
uns bietet.

\begin{figure}[h]
    \centering
    \includegraphics[width= \textwidth]{images/Rest-API.png}
    \caption{Veranschaulichung REST-API}
    \label{fig:beispiel}
    \cite{REST}
\end{figure}

\newpage

\section{Protokoll}
\subsection{1. Woche, Zeitraum 02.10.2023-08.10.2023}
Im Meeting am 06.10.2023 wurden folgenden Sachen hinsichtlich des Projektes
festgelegt:

Quanto Solutions gab uns keine Vorgaben in Hinsicht auf Technologien und
Programmiersprachen, mit der Einschränkung, dass es sich dabei um eine
Webanwendung handeln muss, was sich jedoch auch aus den Vorgaben des
SWTM-Moduls ergab. Somit wurde uns freie Hand gelassen im Aspekt auf die
technischen Entscheidungen. Zudem wurde noch einmal genau auf die Anforderungen
des Projektmanagement Tools eingegangen. Diese wurden unterteilt in Muss-,
Soll- und Kann-Anforderungen. Des Weiteren wurde besprochen, wie ein Projekt
aufgebaut ist und welche verschiedenen Rollen in diesem involviert sind.
Dadurch wurde noch einmal deutlich gemacht, dass jede Rolle in einem Projekt
verschiedene Zugriffsrechte hat. Bis zum nächsten Meeting sollten wir ein
Mockup erstellen.

\subsection{2. Woche Zeitraum 09.10.2023 -22.10.2023}
Im Meeting am 13.10 wurden folgende Sachen hinsichtlich des Projektes
festgelegt.

Zur Zeiterfassung der Mitarbeiterzeiten wird die Start- und Endzeit des
Arbeitstages inklusive Pausenzeiten, manuell von Hand eingetragen und nicht,
wie im ersten Mockup, per Start- und Stop-Button gestoppt. Für das Eintragen
der Arbeitszeiten wurde vereinbart, dass Mitarbeiter ihre Arbeitszeiten auf
Stundenbasis eintragen können und nicht, wie bisher gedacht, nur ganze Tage
buchen können. In Hinsicht auf das nachträgliche Ändern von Arbeitszeiten wurde
festgelegt, dass die Arbeitszeiten bis zum Ende des Monats geändert werden
können. Des Weiteren wurde festgehalten, dass Manager in Projekten alle
Berechtigungen besitzen. Die Projektsprache hinsichtlich der User Interfaces
wurde auf Englisch festgelegt, damit möglichst viele Mitarbeiter diese
verwenden können, hinsichtlich der möglichen Akquisition von internationalen
Firmen. Ebenfalls wurde festgelegt, dass die App-Sidebar auf jeder Seite der
App sichtbar sein soll, um eine möglichst effektive und schnelle Navigation in
der App zu gewährleisten. Hinsichtlich der Planung von Projekten in der
Anwendung wurde festgehalten, dass Manager die Positionen erstellen. Der
Projektleiter kann dann Mitarbeiter für diese Positionen zuordnen und auf
Tagesbasis das Projekt planen.

\subsection{3.Woche Zeitraum 16.10.2023 - 22.1.2023}
Im Meeting am 20.10 wurden folgende Sachen hinsichtlich des Projektes
festgelegt.

Es wurde vom Kunden das blaue Farbschema für das Produkt gewünscht. Zudem wurde
sich auch gewünscht, wenn möglich, die Positionen in “Add Employee” per Drag
and Drop zu veranschaulichen. Doch da haben wir die endgültige Entscheidung,
weswegen wir und auch für das Drag and Drop Feature entschieden haben. Bei
“Manage Project” sollen alle Mitarbeiter angezeigt werden und diese eine
begrenzung von 8 Stunden pro Tag haben. Dabei soll man diese Mitarbeiter direkt
für mehrere Tage, Wochen und Monate einplanen können. Das soll man wiederholen,
zum Beispiel per Checkbox und im Kalender auch rauslöschen können. Eine
Tagessicht wurde ebenfalls vereinbart. In “Time Registration” haben wir
beschlossen das Mitarbeiter am Ende des Monats ihre Zeiten bestätigen können
und diese dann der Projektleiter sehen kann. Die Mitarbeiter können somit frei
im Monat ihre Stunden anpassen. Nach dem bestätigen der Zeiten kann der
Mitarbeiter eine Zeiten nicht mehr ändern, weswegen bei vergessen der
Eintragung, die Zeiten händisch durch den Projektleiter erfolgen müssen.

\subsection{4. Woche Zeitraum: 23.10.203 -29.10.2023}
Im Meeting am 27.10 wurden folgende Sachen hinsichtlich des Projektes
festgelegt.

In Bezug auf unser Kanban-Board wurde festgelegt die User Stories in kleinere
Tasks zu zerlegen. Es wurde zudem festgelegt das wir 2 GitHub Repositories
haben sollen, wobei die erste für das Frontend ist und die zweite für das
Backend. Zur Code Dokumentation wurde gesagt das wir ein einheitlichen
Kommentierstil haben sollen. In Bezug auf das Datenbankmodell sollten ein paar
Änderungen vorgenommen werden, wie die Team Employee Tabelle zu entfernen und
eine TeamId anzulegen. Diese sollten angepasst werden. Es kam die Frage auf für
den SQL Server einen Docker zu verwenden, welches bis zum nächsten Meeting von
der Kundenseite beantwortet werden soll. Der Kalender in “Manage Project” ist
zu viel des Guten und könnte auch durch eine einfache Tabelle dargestellt
werden. Laut dem Kunden ist es unsere Entscheidung zu sagen ob wir ein Kalender
oder eine Tabelle erstellen. Dabei spielt beim einplanen die Uhrzeit keine
Rolle, sondern nur die 8 Stunden. Es wurde vereinbart bis nächste Woche
Dienstag ein MockUp in Form einer Tabelle zu erstellen.

\subsection{5. Woche Zeitraum: 30.10.2023 - 05.11.2023}
Im Meeting am 31.10 wurden folgende Sachen hinsichtlich des Projektes
festgelegt. Zwischen dem Kalender und der Tabelle haben wir uns entschieden
eine Tabelle zu machen und diese mit Vuetify Komponenten und Usability Features
wie Drag and Drop zu erstellen. Dabei soll die Benutzerfreundlichkeit auf
Stundenbasis auch verbessert werden. Hinzu kommt, dass wir bis Freitag, den
03.11, den Mockup abschließen sollen. Es wurde endgültig entschieden, beim
Einplanen der Mitarbeiter per Tabelle, diese für bis zu 8 Stunden einplanen zu
können. Als Zusatzfeature mit geringerer Priorität können wir eine Wiederholung
der Planung erstellen, wo die Planung dann z.B. für jeden Montag der nächsten 2
Monate gilt. Beim Datenmodell wurden einige Tabellen geändert und es wurde
festgelegt das Datenmodell bis zum nächsten Meeting als Bild bereitzustellen,
damit der Kunde sich dazu Gedanken machen kann. Des Weiteren wurde
festgehalten, Docker Container für jeweils das Frontend, Backend sowie für die
Datenbank zu erstellen. Es wurde festgelegt einen zentralen Ort für die
Protokolle zu haben und diese nach dem jedem Meeting mit dem Kunden zu teilen.

Im Meeting am 3.11 wurden folgenden Sachen hinsichtlich des Projektes
festgelegt. Datenbank: Bei der Nutzung und Erstellung der Datenbank sollten
keine Kosten entstehen, sie sollte lizenzfrei sein. Das Projekt ist ein
Open-Source-Projekt. Dokumentation Datenbank: Die SQL Statements sollen ins Git
hinzugefügt werden (Backend) Repository. In der Doku soll das Datenbankmodell
hinzugefügt und erklärt werden. Datenbankmodell: Das Datenbankmodell muss noch
einmal überarbeitet werden. In der Projekttabelle soll folgendes hinzugefügt
werden: Projekt Ersteller, Datum wann das Projekt erstellt wurde, und wer der
Projektmanager ist. In der Tabelle Employees soll das Attribut: team roll auf
title umbenannt werden. Außerdem soll mit passenden Präfixen gearbeitet werden,
zum Beispiel fk bei Foreign Keys. Um die Bezeichnung eindeutig zu machen,
müssen noch alle Attribute, die auf Deutsch sind, auf Englisch umbeannt werden.

Mockup: Zusammen sind wir noch einmal das Mockup durchgegangen und haben letzte
Anpassungen vorgenommen: Es wurde beschlossen, dass nichts dagegen spricht,
dass der Projektleiter auch das Budget sehen kann für seine Projekte. Dem
Manager soll bei der Projekterstellung auch das kalkulierte Budget angezeigt
werden, bevor er das Projekt anlegt. Die Manage Project Ansicht: Soll eine
Tabelle sein in der alle Mitarbeiter aufgelistet sind die diesem Projekt
zugeteilt wurden: Folgendes soll in der Tabelle angezeigt werden: Eine Spalte
mit bereits eingeplante Stunden über alle Projekte des Mitarbeiters dadurch
kann man sehen wie viele Stunden man der Mitarbeiter an dem Tag noch einplanen
könnte. In der nächsten Spalte kann man dann die Anzahl der Stunden eintragen,
wie viele Stunden man den Mitarbeiter an diesem Tag einplanen möchte. Sowie
eine Spalte Positionen, in dem man festlegt, auf welcher Position der
Mitarbeiter an diesem Tag eingeplant werden soll. Diese Liste soll immer einen
kompletten Monat umfassen und man kann per DropDown Menü bequem zwischen den
Monaten hin und her wechseln. Die Working Times Ansicht: Soll auch nun auch in
einer Tabelle dargestellt werden wie die Manage Project Ansicht, für ein
besseres Look and Feel und eine bessere Übersichtlichkeit. Zudem soll es eine
neue Extra-Ansicht geben, in dem der Mitarbeiter seine Zeiten sieht, die er
noch nicht erfasst hat, obwohl er an diesen Tagen eingeplant war. Hier hat er
noch die Möglichkeit, seine Zeiten manuell nachzutragen. Zudem kann der
Mitarbeiter seine Zeiten am Monatsende bestätigen, nachdem er alles
kontrolliert hat. Sobald er Sie bestätigt hat, kann er keine Änderungen mehr
vornehmen. Sobald seine Arbeitszeiten bestätigt wurden, landen diese beim
Projekt Manager, nur er kann jetzt noch Änderungen vornehmen. Bis zum nächsten
Meeting sollten wir mit der Implementierung starten und schauen, dass wir das
Frontend mit dem Backend und der Datenbank verbinden. Um erste Einträge in der
Software vorzunehmen. (Nice to have: Mange Project Ansicht: Mitarbeiter
anklicken um Extra Infos zu erhalten, in welchen anderen Projekten der
Mitarbeiter noch involviert ist.)

\subsection{6. Woche Zeitraum 06.11.2023-12.11.2023}
Teambesprechung mit Andre: Protokoll vom 10.11.2023, 14:00 Uhr bis 15:00 Uhr

Die heutige Teambesprechung mit Andre fokussierte sich auf entscheidende
Aspekte der Authentifizierung, Softwarearchitektur und Meilensteinplanung.
Besonderes Augenmerk wurde auf die SAP-Integration gelegt, wobei
unterschiedliche Berechtigungen je nach Rolle diskutiert wurden. Im Bereich der
Softwarearchitektur standen die Planung von Meilensteinen sowie die
Schnittstellenbeschreibung mittels OpenAPI SWAGGER im Vordergrund. Ein weiterer
Schwerpunkt lag auf der technischen Durchführung und Containerisierung,
einschließlich der Ankündigung von Meilensteinen und dem Backend-Zugriff auf
Datenbanken. Des Weiteren wurde die Vorbereitung für Meilenstein 2
thematisiert, einschließlich der Inhalte und Terminvereinbarungen. Die
Besprechung lieferte einen klaren Fahrplan für die Umsetzung der vereinbarten
Maßnahmen bis zum nächsten Meeting. Teambesprechung mit Lisa: Protokoll vom
10.11.2023, 15:00 Uhr bis 16:00 Uhr

Im Anschluss an die Besprechung mit Andre stand die Teambesprechung mit Lisa im
Fokus. Das Team setzte sich intensiv mit der Gestaltung einer neuen Tabelle in
Figma auseinander. Dabei wurden das Modell und die Ansicht der Tabelle
überprüft und diskutiert. Besondere Aufmerksamkeit galt der Filterung nach
Personen, der Gestaltung der Positionsspalte und der Farbgebung basierend auf
der Stundenanzahl. Die Diskussion über die unabhängige Funktionalität und die
spätere Verbindung rundete die Besprechung ab. Diese Zusammenfassung bietet
einen klaren Überblick über die behandelten Themen und dient als Grundlage für
die Umsetzung der festgelegten Maßnahmen bis zum nächsten Meeting.

\subsection{7. Woche Zeitraum 13.11.2023-19.11.2023}

\newpage

\bibliographystyle{IEEEtran}
\bibliography{quellen}

\end{document}
